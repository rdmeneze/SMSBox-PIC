%%%%%%%%%%%%%%%%%%%%%%%%%%%%%%%%%%%%%%%%%%%%%%%%%%%%%%%%%%%%%%%%%%%%%%%%%%%%%%%%%%%%%%%%%%%%%%%    

\section{Overview}
  
  A plataforma SMS Box permite o envio de mensagens de texto (SMS) atrav�s de uma plataforma 
  de hardware baseada em um processador de 8 bits controlando um modem GSM.

  Este guia foca na utiliza��o do SMS Box como uma plataforma de envio de SMS integrada com a aplica��o GSMComando e a sua integra��o no VosCenter.
    
  \subsection{Placa SMS Box}
    A placa � composta pelos seguintes componentes:
    
    \begin{itemize}
      \item Microcontrolador de uso geral PIC18LF4680 I/P;
      \item Mem�ria flash SDRAM AT45DB041 para download de firmware e gera��o de logs;
      \item Circuito de l�gica de apoio, composta de portas l�gicas e multiplexadores que s�o encarregados 
            de realizar a troca dos chips e conforma��o dos n�veis el�tricos entre o microcontrolador e o modem GSM;
      \item Modem GSM SIMCOM 340C;
      \item LEDs de status de opera��o da m�quina; 
      \item LED de status de conex�o do modem GSM;
      \item Soquetes de SIM Cards; 
      \item Conector de comunica��o serial com PC \textit{Host}; 
      \item Conector de debug do modem GSM.
    \end{itemize}
    